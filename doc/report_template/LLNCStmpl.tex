% LLNCStmpl.tex
% Template file to use for LLNCS papers prepared in LaTeX
%websites for more information: http://www.springer.com
%http://www.springer.com/lncs

\documentclass{llncs}
%Use this line instead if you want to use running heads (i.e. headers on each page):
%\documentclass[runningheads]{llncs}


\begin{document}
\title{Rate-based Synchronous Diffusion}

%If you're using runningheads you can add an abreviated title for the running head on odd pages using the following
%\titlerunning{abreviated title goes here}
%and an alternative title for the table of contents:
%\toctitle{table of contents title}

\subtitle{Internet of Things, Group Project}

%For a single author
%\author{Author Name}

%For multiple authors:
\author{Authors} 


%If using runnningheads you can abbreviate the author name on even pages:
%\authorrunning{abbreviated author name}
%and you can change the author name in the table of contents
%\tocauthor{enhanced author name}

%For a single institute
%\institute{Institute Name \email{email address}}

% If authors are from different institutes 
\institute{University of Bern \email{\{jovana.micic, david.boesiger\}@students.unibe.ch}}

%to remove your email just remove '\email{email address}'
% you can also remove the thanks footnote by removing '\thanks{Thank you to...}'


\maketitle

%\begin{abstract}
%abstract text goes here - Lorem ipsum dolor sit amet, consectetur adipiscing elit, sed do eiusmod tempor incididunt ut labore et dolore magna aliqua.
%\end{abstract}

\section{Protocol Introduction}


Time Synchronization is important in every Wireless Sensor Network, that is because we are interestet in colaborative information. All actions perfomed by the node is controlled by its clock time which may differ from the clock time from other nodes. Since each node has its own clock, we need to synchronize the clock so we are able to get consitent information. Distributed synchronization protocols are required to coordinate the nodes in the network so that they
follow the same reference frame. \textbf{Rate-Based Diffusion Protocol (RDP)} aims to synchronize the nodes in the network to the average value of the clocks in the
network. Instead of the timing information, the difference between the clocks of
the nodes and their relative importance is diffused in the network. Operation of the diffusion protocol is based on two things: comparison of the local clocks of two nodes
and adjusting the clocks accordingly.

\section{Methods}

\section{Experimental setup/Measurement procedure}

\section{Results and Analysis}

\section{Conclusions}


%The bibliography, done here without a bib file
%This is the old BibTeX style for use with llncs.cls
\bibliographystyle{splncs}

%Alternative bibliography styles:
%the following does the same as above except with alphabetic sorting
%\bibliographystyle{splncs_srt}
%the following is the current LNCS BibTex with alphabetic sorting
%\bibliographystyle{splncs03}
%If you want to use a different BibTex style include [oribibl] in the document class line

\begin{thebibliography}{1}
%add each reference in here like this:
\bibitem[RE1]{reference1}
Author:
Article/Book:
Other info: (date) page numbers.
\end{thebibliography}

\end{document}

